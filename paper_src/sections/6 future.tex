\section{Future Research}
\label{sec:future}


Based on the results of the experiment leaves room for both promise, and plenty of future work. Plenty of these results showed that there can be success, and the measure of success being close to a 50 percent win ratio, with this approach, however this model
is simplistic even if it already shows promise.

Work to be done in the future on this research would firstly be, a more complicated environment and model space specifically for Starcraft II, but even further then that, a multitude of games. In theory, if it works in a game like Starcraft II with actions and 
environment states on an increasingly high level, it should also be applicable to other games. Using other models research that was mentioned here\cite{starcraft_unplugged}\cite{liu2021mAS}\cite{liu2021mASreport}\cite{vinyals2017starcraft}, and many more as the 
research in creating a near perfect skilled model in a video games, like Starcraft II, has been thoroughly and expertly researched. Exploiting that research and combining this with more computation capabilities and time into the problem would lead to a potential
agent whose skill truly could compete with any level and not just a few basic models. 

Along with that, further research could be done in making the skill level alternate, for example one out of ever three actions could be the best, or fluctuating between half best, second best, and so on. Other methodologies on how to fluctuate skill could also 
be researched and tested  as this was simple a proof of concept and first go around of testing a complicated environment with the intention of making a model not perfect all the time, but the novelty being changing its skill level which was proving partially true
capable of doing so in a simple setting. 

Further research in developing an agent that players can select the level of challenge, for example neutral without alternating above in the DQN, managed around a 40 percent win ratio against each opponent. So if a player wanted a more relaxed but not easy gameplay
this would be a good difficulty setting to apply, adding further novelty to this idea with not only scaling difficulties, but indeed the capability to determine how challenging a model can be based on the players skill. 